\documentclass[12pt]{article}
\usepackage[hidelinks]{hyperref}
\usepackage[square,sort,comma,numbers]{natbib}
\title{EECS 493 Progress Report}
\author{
    Andrew Mason\\
    \texttt{ajm188@case.edu}
    \and
    Jon Pfeil\\
    \texttt{jwp69@case.edu}
}
\date{October 27, 2014}
\begin{document}
\maketitle
\tableofcontents
\pagebreak

\section{Introduction}
Software verification, the ability to prove the correctness of a piece of software \cite{FormalSoftwareVerification}, is an appealing approach to improving software quality. However, the requirement of having formal descriptions of correctness has prevented the widespread use of software verification \cite{MiningSpecifications}. \textit{Specification Mining}, as described in \cite{MiningSpecifications}, is the machine learning approach to discovering formal specifications for a program interface or an abstract data type. A specification mining tool would reduce the major barrier to program verification by automating the generation of formal specifications.

Our project aims to push forward the current state of specification mining through the application of frequent subgraph mining algorithms to novel intermediate representations of programs.

\section{Background}
\label{section:Background}
\subsection{Formal Verification}
\textit{expand for final report}

\subsection{Intermediate Representations}
\textit{expand for final report}

\subsection{Program Dependency Graphs}
\textit{expand for final report}

\subsection{gSpan}
\label{subsection:gSpan}
gSpan is an algorithm for frequent graph-based pattern mining. It explores depth-first  search (DFS) in frequent subgraph mining using a DFS lexicographical ordering and minimum DFS codes, which form a canonical labeling system to support the DFS search \cite{Yan:2002:GGS:844380.844811}. The algorithm exploits the fact that, given a set of DFS codes for a graph, two graphs are isomorphic if and only if they have the same minimum DFS code.

\section{Related Work}
\subsection{GMFMM and IHMM}
GMFMM and IHMM are two algorithms proposed by Chang and Podgurski for discovering both intraprocedural and interprocedural program dependencies. The algorithms model rules as graph minors of augmented dependence graphs, with edges to indicate shared data dependencies \cite{DBLP:journals/smr/ChangP12}.

The first, GMFMM, is a greedy algorithm for rule discovery by mining maximal frequent minors from interprocedural dependence spheres \cite{DBLP:journals/smr/ChangP12}. The algorithm operates in two phases - the first finds nodes in frequent minors, and the second makes those frequent minors maximal by adding additional dependencies to the graph.

IHMM is the incremental heuristic minor matching algorithm, which discovers violations of rules, even if those instances are interprocedural. Given a user-specified key node \textit{n} (usually a call-site node), IHMM searches \textit{n}'s base function and, if necessary, any ancestors of the base function to identify rule violations involving \textit{n} \cite{DBLP:journals/smr/ChangP12}. IHMM also uses dependence spheres for finding correct instances and violations of programming rules. If IHMM cannot find a correct instance of the rule at the base function (by looking for an isomorphic minor), it looks to the ancestor functions, where it decides that a dependence sphere \textit{S} contains a violation if the number of super-spheres containing no rule instance is greater than some user-specified threshold \cite{DBLP:journals/smr/ChangP12}.

These algorithms are limited in that they will not detect rules whose elements are not connected by a data, control, or shared data dependency. The algorithms are also limited to looking within a given size bound for the dependence sphere \cite{DBLP:journals/smr/ChangP12}.

\section{Proposed Work}
The goal of this project is to push forward the state of specification mining research. This will be done by applying frequent subgraph mining algorithms to Program Dependency Graphs that are generated via \hyperref[subsection:JPDG]{JPDG} run on java source code. There are two measures of success for this goal:

\begin{enumerate}
    \item Improve upon the speed of frequent subgraph mining on program dependency graphs with a baseline given by gSpan
    \item Increase the number of non-trivial specifications mined over baseline specification mining systems.
\end{enumerate}

\noindent To accomplish these goals, the specific work items must be done:

\begin{enumerate}
    \item Complete literature surveys of specification mining and frequent subgraph mining.
    \item Get a working implementation of \hyperref[subsection:JPDG]{JPDG} to generate PDGs from Java source code.
    \item Get a working implementation of the gSpan algorithm and apply it out-of-box to PDGs generated with JPDG.
    \item Implement baseline specification mining systems.
    \item Setup a testing framework to compare the speed of our mining algorithms and the number of non-trivial specifications mined.
    \item Experiment with novel intermediate representations that will allow for specification mining across various levels of abstraction.
\end{enumerate}

\section{Completed Work and Next Steps}

\subsection{Completed Work}
To date, the following work items have been completed for this project:

\begin{enumerate}
    \item Started literature surveys of specification mining and frequent subgraph mining.
    \item Repaired \hyperref[subsection:JPDG]{JPDG} to a working state. All of \hyperref[subsection:JPDG]{JPDG's} unit tests now pass.
    \item Ran \hyperref[subsection:JPDG]{JPDG} on a non-trivial Java project (\href{https://github.com/BonzaiThePenguin/WikiSort}{WikiSort}) and obtained a serialized program dependency graph as output.
    \item Defined more specific hypotheses to investigate in our project proposal
    \item Added a \hyperref[subsection:Docker]{Docker} subsection to the \hyperref[section:Tools and Data Sources]{Tools and Data Sources}.
\end{enumerate}

\subsection{Next Steps}
Based on the work we have completed thus far, the following work should be completed (in addition to our original proposed work):

\begin{enumerate}
    \item Create a \hyperref[subsection:Docker]{docker} image in which to run JPDG. Getting a working version of JPDG involved a tremendous amount of system and depency configuration as well as making subtle tweaks to the scripts that run JPDG. Having a \hyperref[subsection:Docker]{docker} container that has all of this done already would save other users an enormous amount of effort.
    \item Fill in incomplete subsections in the \hyperref[section:Background]{background} section.
\end{enumerate}

\section{Methodology and Evaluation}
The following is an outline of the methodology we will use to evaluate our specification miner:

\begin{enumerate}
    \item Obtain a corpus of programs that already have their non-trivial specifications identified (this can be done by hand if we cannot obtain rights to a previously used data set)
    \item Run \hyperref[subsection:gSpan]{gSpan} on the PDG generated by \hyperref[subsection:JPDG]{JPDG} for each example program. Record the runtime as our speed baseline
    \item Run multiple other specification miners on the program corpus. Record the number of non-trivial specifications mined by each miner on each example. These results will serve as our mining baselines
    \item For each experimental specification miner we want to evaluate:
    \begin{enumerate}
        \item Run on each example in the program corpus
        \item Record the runtime
        \item Record the number of non-trivial specifications mined
    \end{enumerate}
\end{enumerate}

\section{Tools and Data Sources}
\label{section:Tools and Data Sources}
\subsection{Soot}
Soot is a Java optimization framework, intended to be used as either a stand-alone tool to inspect class files, or to develop optimizations or transformations on Java byte code. Soot provides several different intermediate representations of Java byte code: \cite{lam11:_soot_java}
\begin{itemize}
    \item\textbf{Baf}: a streamlined representation of bytecode which is simple to manipulate.
    \item\textbf{Jimple}: a typed 3-address intermediate representation suitable for optimization. This is the primary representation used by Soot.
    \item\textbf{Shimple}: an SSA variation of Jimple.
    \item\textbf{Grimp}: an aggregated version of Jimple suitable for decompilation and code inspection.
    \item\textbf{Dava}: an abstract syntax tree-based representation. Produced via decompilation of the Jimple representation.
\end{itemize}
Intraprocedurally, the Soot framework is often used for its support for implementing intraprocedural data-flow analyses. Soot also provides call graph information for interprocedural analysis as part of its output. \cite{lam11:_soot_java}

\subsection{Heros}
Heros is a an interprocedural, finite, distributive subset (IFDS) problem solving framework, that can be plugged into Java-based program analysis frameworks, like Soot.

\subsection{Jasmin}
Jasmin is a back end for the Soot framework. It is used to compile Soot from source.

\subsection{Smali}
Smali is an assembler and disassembler for the JVM. Additonally, it fully supports the full functionality of the dex format (Android executable files).

\subsection{Parsemis}
% From: https://www2.cs.fau.de/EN/research/zold/ParSeMiS/index.html
ParSeMiS is the parallel and sequential mining suite from the Friedrich-Alexander Universit{\"a}t Erlangen-N{\"u}remburg. It utilizes parallel or specialized algorithms or heuristics to search for frequent, interesting substructures in graph datasets.

\subsection{JPDG}
\label{subsection:JPDG}
JPDG is a tool developed by Tim Henderson which utilizes the Soot framework to generate procedure dependence graphs (PDGs) for Java programs. The PDGs can then be mined for frequent subgraphs to infer programming rules from source code.

\subsection{Google Code}
We will be using a collection of thousands of Java programs collected from Google Code by Tim Henderson for evaluating our research.

\subsection{Docker}
\label{subsection:Docker}
\href{http://www.docker.com}{Docker} is an open-source container virtualization platform, which allows applications to be run inside a specific environment (the container). Docker images can be checked in to projects. Once a docker image has been checked in, everyone working on the process can create a docker container based on this image. This guarantees everyone working on a project is working on it in identical environments, alleviating any issues caused by system dependencies.

\pagebreak
\bibliographystyle{acm}
\bibliography{proposal}
\end{document}