\documentclass[12pt]{article}
\usepackage[pdftex]{hyperref}
\title{EECS 493 Research Proposal}
\author{
    Andrew Mason\\
    \texttt{ajm188@case.edu}
    \and
    Jon Pfeil\\
    \texttt{jwp69@case.edu}
}
\date{September 26, 2014}
\begin{document}
\maketitle
\tableofcontents
\pagebreak
\section{Abstract}
\section{Project Goals}
\section{Tools}
\subsection{Soot}
% Info taken from: http://www.sable.mcgill.ca/soot/
% and: http://www.sable.mcgill.ca/publications/papers/2011-6/11.cetus.soot.pdf
Soot is a Java optimization framework, intended to be used as either a stand-alone tool to inspect class files, or to develop optimizations or transformations on Java byte code. Soot provides several different intermediate representations of Java byte code:
\begin{itemize}
    \item\textbf{Baf}: a streamlined representation of bytecode which is simple to manipulate.
    \item\textbf{Jimple}: a typed 3-address intermediate representation suitable for optimization. This is the primary representation used by Soot.
    \item\textbf{Shimple}: an SSA variation of Jimple.
    \item\textbf{Grimp}: an aggregated version of Jimple suitable for decompilation and code inspection.
    \item\textbf{Dava}: an abstract syntax tree-based representation. Produced via decompilation of the Jimple representation.
\end{itemize}
Intraprocedurally, the Soot framework is often used for its support for implementing intraprocedural data-flow analyses. Soot also provides call graph information for interprocedural analysis as part of its output.
\subsection{Heros}
Heros is a an interprocedural, finite, distributive subset (IFDS) problem solving framework, that can be plugged into Java-based program analysis frameworks, like Soot.
\subsection{Jasmin}%not really sure what this does
\subsection{Smali}%also not really sure about this one either
\subsection{Parsemis}
% From: https://www2.cs.fau.de/EN/research/zold/ParSeMiS/index.html
ParSeMiS is the parallel and sequential mining suite from the Friedrich-Alexander Universit{\"a}t Erlangen-N{\"u}remburg. It utilizes parallel or specialized algorithms or heuristics to search for frequent, interesting substructures in graph datasets.
\subsection{JPDG}
JPDG is a tool developed by Tim Henderson which utilizes the Soot framework to generate procedure dependence graphs (PDGs) for Java programs. The PDGs can then be mined for frequent subgraphs to infer programming rules from source code.
\section{Data Sources}
\end{document}