\documentclass[12pt]{article}
\usepackage[pdftex]{hyperref}
\title{EECS 493 Research Proposal}
\author{Andrew Mason \and Jon Pfeil}
\date{September 26, 2014}
\begin{document}
\maketitle
\tableofcontents
\pagebreak
\section{Abstract}
\section{Project Goals}
\section{Tools}
\subsection{Soot}
% Info taken from: http://www.sable.mcgill.ca/soot/
% and: http://www.sable.mcgill.ca/publications/papers/2011-6/11.cetus.soot.pdf
Soot is a Java optimization framework, intended to be used as either a stand-alone tool to inspect class files, or to develop optimizations or transformations on Java byte code. Soot provides four different intermediate representations of Java byte code:
\begin{itemize}
    \item\textbf{Baf}: a streamlined representation of bytecode which is simple to manipulate.
    \item\textbf{Jimple}: a typed 3-address intermediate representation suitable for optimization. This is the primary representation used by Soot.
    \item\textbf{Shimple}: an SSA variation of Jimple.
    \item\textbf{Grimp}: an aggregated version of Jimple suitable for decompilation and code inspection.
\end{itemize}
\subsection{Heros}
\subsection{Jasmin}%not really sure what this does
\subsection{Smali}%also not really sure about this one either
\subsection{Parsemis}
% From: https://www2.cs.fau.de/EN/research/zold/ParSeMiS/index.html
ParSeMiS is the parallel and sequential mining suite from the Friedrich-Alexander Universit{\"a}t Erlangen-N{\"u}remburg. It utilizes parallel or specialized algorithms or heuristics to search for frequent, interesting substructures in graph datasets.
\subsection{JPDG}
\section{Data Sources}
\end{document}